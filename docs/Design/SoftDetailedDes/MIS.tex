\documentclass[12pt, titlepage]{article}

\usepackage{amsmath, mathtools}

\usepackage[round]{natbib}
\usepackage{amsfonts}
\usepackage{amssymb}
\usepackage{graphicx}
\usepackage{colortbl}
\usepackage{xr}
\usepackage{hyperref}
\usepackage{longtable}
\usepackage{xfrac}
\usepackage{tabularx}
\usepackage{float}
\usepackage{siunitx}
\usepackage{booktabs}
\usepackage{multirow}
\usepackage[section]{placeins}
\usepackage{caption}
\usepackage{fullpage}
\usepackage[normalem]{ulem}   
\usepackage{xcolor}

\hypersetup{
bookmarks=true,     % show bookmarks bar?
colorlinks=true,       % false: boxed links; true: colored links
linkcolor=red,          % color of internal links (change box color with linkbordercolor)
citecolor=blue,      % color of links to bibliography
filecolor=magenta,  % color of file links
urlcolor=cyan          % color of external links
}

\usepackage{array}

\externaldocument{../../SRS/SRS}

% %% Comments

\usepackage{color}

\newif\ifcomments\commentstrue %displays comments
%\newif\ifcomments\commentsfalse %so that comments do not display

\ifcomments
\newcommand{\authornote}[3]{\textcolor{#1}{[#3 ---#2]}}
\newcommand{\todo}[1]{\textcolor{red}{[TODO: #1]}}
\else
\newcommand{\authornote}[3]{}
\newcommand{\todo}[1]{}
\fi

\newcommand{\wss}[1]{\authornote{blue}{SS}{#1}} 
\newcommand{\plt}[1]{\authornote{magenta}{TPLT}{#1}} %For explanation of the template
\newcommand{\an}[1]{\authornote{cyan}{Author}{#1}}

% %% Common Parts

\newcommand{\progname}{ProgName} % PUT YOUR PROGRAM NAME HERE
\newcommand{\authname}{Team \#, Team Name
\\ Student 1 name and macid
\\ Student 2 name and macid
\\ Student 3 name and macid
\\ Student 4 name and macid} % AUTHOR NAMES                  

\usepackage{hyperref}
    \hypersetup{colorlinks=true, linkcolor=blue, citecolor=blue, filecolor=blue,
                urlcolor=blue, unicode=false}
    \urlstyle{same}
                                


\begin{document}

\title{\textbf{yoGERT GIS Toolbox}\\ Capstone 4G06\\Module Interface Specification for \progname{yoGERT}}

\author{Team 19,
		\\ Smita Singh, Abeer Alyasiri, Niyatha Rangarajan,\\ Moksha Srinivasan, Nicholas Lobo, Longwei Ye \\\\
}

\date{\today}

\maketitle

\pagenumbering{roman}

\section{Revision History}

\begin{tabularx}{\textwidth}{p{3cm}p{2cm}X}
\toprule {\bf Date} & {\bf Version} & {\bf Notes}\\
\midrule
January 18, 2023 & 1.0 & \textbf{Smita}:Generating Activity Locations Module, Main Module, Data Transformation Module. \textbf{Abeer}:Network Graph, Shortest Route, Alternative Route, Route Generation, Mapping. \textbf{Moksha}: Preprocessing, Data Transformation. \textbf{Longwei} Module Decomposition, \textbf{Niyatha} Generating episodes \\
February 5th, 2023 & 1.1 & \textbf{Smita}: Updating Modules, addressing issues #51 and #52 \textbf{Moksha}: Updating Preprocessing Modules, addressing issues #50 and #53 from Team 14's Review \\
\bottomrule
\end{tabularx}

~\newpage

\section{Symbols, Abbreviations and Acronyms}

See \href{https://github.com/NicLobo/Capstone-yoGERT/blob/main/docs/SRS/SRS.pdf}{SRS} 

\newpage

\tableofcontents

\newpage

\pagenumbering{arabic}

\section{Introduction}

The following document details the Module Interface Specifications for
\wss{the yoGERT toolbox}

Complementary documents include the \href{https://github.com/NicLobo/Capstone-yoGERT/blob/main/docs/SRS/SRS.pdf}{System Requirement Specification}
and \href{https://github.com/NicLobo/Capstone-yoGERT/blob/main/docs/Design/SoftArchitecture/MG.pdf}{Module Guide}. The full documentation and implementation can be
found at \url{https://github.com/NicLobo/Capstone-yoGERT}.  %\wss{provide the url for your repo}

\section{Notation}
The structure of the MIS for modules comes from \citet{HoffmanAndStrooper1995},
with the addition that template modules have been adapted from
\cite{GhezziEtAl2003}.  The mathematical notation comes from Chapter 3 of
\citet{HoffmanAndStrooper1995}.  For instance, the symbol := is used for a
multiple assignment statement and conditional rules follow the form $(c_1
\Rightarrow r_1 | c_2 \Rightarrow r_2 | ... | c_n \Rightarrow r_n )$.

The following table summarizes the primitive data types used by \progname. 

\begin{center}
\renewcommand{\arraystretch}{1.2}
\noindent 
\begin{tabular}{l l p{7.5cm}} 
\toprule 
\textbf{Data Type} & \textbf{Notation} & \textbf{Description}\\ 
\midrule
character & char & a single symbol or digit\\
integer & $\mathbb{Z}$ & a number without a fractional component in (-$\infty$, $\infty$) \\
natural number & $\mathbb{N}$ & a number without a fractional component in [1, $\infty$) \\
real & $\mathbb{R}$ & any number in (-$\infty$, $\infty$)\\
\bottomrule
\end{tabular} 
\end{center}

\noindent
The specification of \progname \ uses some derived data types: sequences, strings, and
tuples. Sequences are lists filled with elements of the same data type. Strings
are sequences of characters. Tuples contain a list of values, potentially of
different types. In addition, \progname \ uses functions, which
are defined by the data types of their inputs and outputs. Local functions are
described by giving their type signature followed by their specification.

\newpage
\section{Module Decomposition}

The following table is taken directly from the \href{}{Module Guide} document for this project.

\begin{table}[h!]
\centering
\begin{tabular}{p{0.3\textwidth} p{0.6\textwidth}}
\toprule
\textbf{Level 1} & \textbf{Level 2}\\
\midrule

{Hardware-Hiding} & ~ \\
\midrule

\multirow{7}{0.3\textwidth}{Behaviour-Hiding} & Generate Episode\\
& Route Generation\\
& Fetch Activity Locations\\
& Mapping\\
& Main\\ 
\midrule

\multirow{3}{0.3\textwidth}{Software Decision} & Preprocessing Inputs\\
& Network Graph\\
& Shortest Route\\
& Alternative Route\\
& Data Transformation\\

\bottomrule

\end{tabular}
\caption{Module Hierarchy}
\label{TblMH}
\end{table}

\newpage

\section{MIS of \wss{Module Preprocessing}} \label{Module} 

\subsection{Module}
Preprocessing Inputs

\subsection{Uses}

\subsection{Syntax}

\subsubsection{Exported Constants}

\subsubsection{Exported Access Programs}

\begin{center}
\begin{tabular}{p{3cm} p{4cm} p{4cm} p{3cm}}
\hline
\textbf{Name} & \textbf{In} & \textbf{Out} & \textbf{Exceptions} \\
\hline
\wss{validateCSV} & CSV & $\mathbb{B}$, CSV & $\mbox{\textcolor{red}{InvalidInput}}$ $\mbox{\textcolor{red}{IOError}}$ \\
\hline
\end{tabular}
\end{center}

\subsection{Semantics}

\subsubsection{State Variables}

None.

\subsubsection{Environment Variables}

IOmanager : it is the access point between the local memory and the 
application. It is used when saving files locally.

\subsubsection{Assumptions}

\begin{itemize}
    \item Assume CSV file paths are valid when working within the same directory. 
\end{itemize}

\subsubsection{Access Routine Semantics}

\noindent \wss{validateCSV}(csvfile):
\begin{itemize}
\item transition: \wss{None} 
\item output: \wss{\textcolor{red}{(\forall columns : str | file \in validateCSV(csvfile) : validateCols(columns)) $\Rightarrow$ \emph{true}}}
\item output: \wss{\textcolor{red}{$(\forall rows : tuple($\mathbb{R}, \mathbb{R}, str$) | file \in normalizeCSV(csvfile) : validateData(rows))) \Rightarrow \mbox{CSV})$}}

\item exception: \wss{$(\neg(((\forall columns : str | file \in validateCSV(csvfile) : validateCols(columns)) $\Rightarrow$ \mbox{InvalidInput}$} 

\item exception: \wss{\textcolor{red}{$(\neg(\forall rows : tuple($\mathbb{R}, \mathbb{R}, str$) | file \in normalizeCSV(csvfile) : validateData(rows))) \Rightarrow \mbox{IOError})$}} 
\end{itemize}

\subsubsection{Local Functions}

\textcolor{red}{normalizeCSV : removes any invalid gps and time points from the original CSV data}
\textcolor{red}{validateCols : Ensures that data has latitude, longitude, and time columns}

\newpage

\section{MIS of \wss{Module Network Graph}} \label{Module} 

\subsection{Template Module}
Network Graph

\subsection{Uses}
None.

\subsection{Syntax}

\subsubsection{Exported Constants}
DISTANCETOLERANCE = 200 - tolerance distance (m) added to the radius of the circle for the mapped area that encapsulates all the input GPS coordinates. \\
EARTHRADIUS = 6371000 - earth's radius (m).

\subsubsection{Exported Type}
NetworkGraph = ?

\subsubsection{Exported Access Programs}

\begin{tabular}{p{3cm} p{4cm} p{4cm} p{4cm}}
\hline
\textbf{Name} & \textbf{In} & \textbf{Out} & \textbf{Exceptions} \\
\hline
\wss{new NetworkGraph} & tuple of (latitude: $\mathbb{R}$, longitude: $\mathbb{R}$), tuple of (latitude: $\mathbb{R}$, longitude: $\mathbb{R}$), seq of tuple of (latitude: $\mathbb{R}$, longitude: $\mathbb{R}$), seq of \{drive, walk, bike\} & NetworkGraph & \mbox{InvalidMode} \\
\hline
\wss{getNearestNode} & tuple of (latitude: $\mathbb{R}$, longitude: $\mathbb{R}$) & $\mathbb{N}$ & \mbox{OutOfBoundsCoord} \\
\hline
\end{tabular}

\subsection{Semantics}

\subsubsection{State Variables}

\emph{dist} : $\mathbb{R}$ - distance in m for the radius of the network graph using GPS coordinates. \\
\emph{stcoord} : tuple of (latitude: $\mathbb{R}$, longitude: $\mathbb{R}$) - starting GPS coordinates inputted by the user. \\
\emph{nodes} : seq of $\mathbb{N}$ - network graph node. \\
\emph{edges} : seq of tuple of ($\mathbb{N}$, $\mathbb{N}$, $\mathbb{R}$) - network graph edge defined by 2 nodes and weight extracted from.\\


\subsubsection{Environment Variables}

onlinenetworkdatabase : connection to online database to retrieve information layers of intersections, roads, and paths initialised within module to build the network graph.  

\subsubsection{Assumptions}
\begin{itemize}
    \item Assume the inputted GPS coordinates are valid as they are the output of another module.
    \item Assume the online network database is always accessible within 10 minutes.
\end{itemize}

\subsubsection{Access Routine Semantics}

\noindent \wss{new NetworkGraph}(\emph{startcoord, endcoord, stopcoords, networkmode}):
\begin{itemize}
\item transition: \wss{The \emph{dist} is set by $finddistance(startcoord, endcoord, stopcoords)$ which is the largest distance between $startcoord$ and any other coordinate given in the input. \emph{stcoord} is set to \emph{startcoord} and sets the rest of the state variables using extracted data from \emph{onlinenetworkdatabase} upon creation of the object \emph{NetworkGraph}.} 
\item output: \wss{Creates a \emph{NetworkGraph} object with the parameters \emph{startcoord, endcoord, stopcoords, networkmode} and extracted data from \emph{onlinenetworkdatabase}} 
\item exception: \wss{$(\neg(networkmode \in \{drive, walk, bike\}) \Rightarrow \mbox{InvalidMode}))$} 
\end{itemize}

\noindent \wss{getNearestNode}(\emph{coord}):
\begin{itemize}
\item transition: \wss{None} 
\item output: \textcolor{red}{\wss{nodes[i] where $i \in [0..|stopcoords|-1] $} }
\item exception: \wss{$(\neg(findhdistance(coord,stcoord) \leq \mbox{dist}) \Rightarrow \mbox{OutOfBoundsCoord}))$}
\end{itemize}




%end of shortest
\subsubsection{Local Functions}

finddistance : tuple of (latitude: $\mathbb{R}$, longitude: $\mathbb{R}$) $\times$ tuple of (latitude: $\mathbb{R}$, longitude: $\mathbb{R}$) $\times$ seq of tuple of (latitude: $\mathbb{R}$, longitude: $\mathbb{R}$) $\rightarrow \mathbb{R}$
\begin{itemize}
    \item Description: Computes largest distance using \emph{findhdistance} between starting coordinate and another coordinate, either a destination coordinate or a stop coordinate.
\end{itemize}
findhdistance : tuple of (latitude: $\mathbb{R}$, longitude: $\mathbb{R}$) $\times$ tuple of (latitude: $\mathbb{R}$, longitude: $\mathbb{R}$) $\rightarrow \mathbb{R}$
\begin{itemize}
    \item Description: finds the distance between two GPS coordinates using Haversine formula. 
\end{itemize}


\newpage



\section{MIS of \wss{Generate Episodes}} \label{Module} 

\subsection{Template Module}
Generate Episodes

\subsection{Uses}
None.

\subsection{Syntax}

\subsubsection{Exported Constants}
N/A

\subsubsection{Exported Type}
Activityepisode = ?

\subsubsection{Exported Access Programs}

\begin{tabular}{p{2cm} p{3cm} p{4cm} p{4cm}}
\hline
\textbf{Name} & \textbf{In} & \textbf{Out} & \textbf{Exceptions} \\
\hline
\wss{new Activityepisode} & A full path to a .csv file containing GPS points (Start: Lat, Long, Time, Stop: Lat, Long, Time) & A full path to a .csv file containing preprocessed GPS data & invalidCSV \\ 
\hline

\end{tabular}

\subsection{Semantics}

\subsubsection{State Variables}
\emph{points} : columns of csv containing ($\mathbb{R}$, $\mathbb{R}$, ${datetime}$) - latitude, longitude and timestamp for a point.\\
\emph{finalpoints} : columns of csv containing ($\mathbb{R}$, $\mathbb{R}$, ${datetime}$) - latitude, longitude and timestamp for a point after data has been run with generateEpisodes(csvPath).\\
\emph{mode} : enum of ($\mathbb{N}$, $\mathbb{N}$, $\mathbb{N}$) - modes for episodes, STOP = 0, WALK = 1, DRIVE = 10\\
s
\subsubsection{Environment Variables}
N/A
\subsubsection{Assumptions}
\begin{itemize}
    \item Assume the values for latitude, longitude and time in the given file are valid.
\end{itemize}

\subsubsection{Access Routine Semantics}

\noindent \wss{new Activityepisode}(\emph{csvPath}):
\begin{itemize}
\item transition: A full path to a .csv file containing GPS points (Start: Lat, Long, Time, Stop: Lat, Long, Time) to a full path to a .csv file containing preprocessed GPS data.
\item output: \wss{finalpoints[i] where $i \in [0..|len(points)|-1]$
\item exception: \wss{$(\neg(csvPath) \Rightarrow \mbox{InvalidFile}))$}
\end{itemize}


\noindent \wss{createSegments}(\emph{csvPath, title}):
\begin{itemize}
\item transition: \wss{The given csv file is parsed and trimmed based using the average stepsize of the given data points.} 
\item output: \wss{points[i] where $i \in [0..|len(points)|-1]$ $ \wedge $$ i = i + (points[-1]-points[0])/len(points)$} 
\item exception: \wss{$(\neg(csvPath) \Rightarrow \mbox{InvalidFile}))$} 
\end{itemize}


\noindent \wss{createVelocities}(\emph{csvPath}):
\begin{itemize}
\item transition: \wss{The given csv file is used to create another csv file with the added column velocity} 
\item output: \wss{points.append(velocity[i] where $(sqrt((i.lat -(i+1).lat)^2+(i.long -(i+1).long)^2)/(i.time - (i+1).time)$} 
\item exception: \wss{$(\neg(csvPath) \Rightarrow \mbox{InvalidFile}))$} 
\end{itemize}


\noindent \wss{generateEpisodes}(\emph{csvPath}):
\begin{itemize}
\item transition: \wss{Generate activity episodes for each mode change in \emph{points} in the csv file.} 
\item output: \wss{endMode[i] where $(( velocity[i] \>= mode.WALK.value and velocity[i] \< mode.DRIVE.value) \wedge endMode = mode.WALK) \vee (endVel \>= mode.DRIVE.value \wedge endMode = mode.DRIVE)\vee endMode = mode.STOP$
} 
\item exception: \wss{$(\neg(csvPath) \Rightarrow \mbox{InvalidFile}))$} 
\end{itemize}

\newpage


\section{MIS of \wss{Data Transformation Module}} \label{Module} 

\subsection{Module}

Data Transformation Module

\subsection{Uses}
None

\subsection{Syntax}

\subsubsection{Exported Constants}

\subsubsection{Exported Access Programs}

\begin{center}
\begin{tabular}{p{4cm} p{4cm} p{4cm} p{4cm}}
\hline
\textbf{Name} & \textbf{In} & \textbf{Out} & \textbf{Exceptions} \\
\hline
\wss{GenInput} & A full path to a .csv containing travel episodes with mode detected & seq of tuple of (latitude: $\mathbb{R}$, longitude: $\mathbb{R}$) & invalidCSV \\
\hline

\end{tabular}
\end{center}

\subsection{Semantics}
\subsubsection{State Variables}
None
\subsubsection{Environment Variables}
None
\subsubsection{Assumptions}
We assume that the input files are generated from the GenerateEpisodes module.

\subsubsection{Access Routine Semantics}


\noindent \wss{GenInput}(travelEpisodes):
\begin{itemize}
\item transition: N/A
\item output: collection of tuple of (latitude: $\mathbb{R}$, longitude: $\mathbb{R}$) provided by genHelper
\item exception: \wss{invalidCSV} 
\end{itemize}


\subsubsection{Local Functions}


genHelper : Path of CSV file(String)
\begin{itemize}
    \item Description : Parses file path and reads CSV file content and converts content and returns seq of tuple of $(\mathbb{R}, \mathbb{R})  \rightarrow tuple of (\mathbb{R}, \mathbb{R})$
\end{itemize}


  
\newpage

\section{MIS of \wss{Module Shortest Route}} \label{ModuleSPath} 

\subsection{Template Module}
Shortest Route

\subsection{Uses}%what modules it uses 
Network Graph

\subsection{Syntax}

\subsubsection{Exported Constants}
None.

\subsubsection{Exported Type}
ShortestRoute = ?

\subsubsection{Exported Access Programs}

\begin{tabular}{p{3cm} p{5cm} p{4cm} p{4cm}}
\hline
\textbf{Name} & \textbf{In} & \textbf{Out} & \textbf{Exceptions} \\
\hline
\wss{new ShortestRoute} & NetworkGraph, tuple of (latitude: $\mathbb{R}$, longitude: $\mathbb{R}$), tuple of (latitude: $\mathbb{R}$, longitude: $\mathbb{R}$), seq of \{time, distance\} & ShortestRoute & $\mbox{InvalidWeight}$, $\mbox{OutOfBoundsCoord}$ \\
\hline
\end{tabular}


\subsection{Semantics}

\subsubsection{State Variables}

\emph{startnode} : $\mathbb{N}$ - graph node nearest the starting GPS coordinate.\\
\emph{endnode} : $\mathbb{N}$ - graph node nearest the destination GPS coordinate.

\subsubsection{Environment Variables}

None.

\subsubsection{Assumptions}

\begin{itemize}
    \item Assume the inputted GPS coordinates are valid as they are the output of another module.
\end{itemize}

\subsubsection{Access Routine Semantics}

\noindent \wss{new ShortestRoute}(\emph{graph, startcoord, endcoord, weighttype}):
\begin{itemize}
\item transition: \wss{Set state variables  $startnode, endnode := findnode(graph, startcoord)$, $findnode(graph, endcoord) $} 
\item output: \wss{Creates \emph{ShortestRoute} using \emph{DijkstraAlg(graph, weighttype, startnode, endnode)}} 
\item exception: \wss{$(\neg(weighttype \in \{time, distance\}) \Rightarrow \mbox{InvalidWeight}) \vee (\neg(startnode \in seq of graph.nodes) \Rightarrow \mbox{OutOfBoundsCoord}) \vee (\neg(endnode \in seq of graph.nodes) \Rightarrow \mbox{OutOfBoundsCoord})$} 
\end{itemize}

\subsubsection{Local Functions}

findnode : \emph{NetworkGraph} $\times$ tuple of $(\mathbb{R}, \mathbb{R})  \rightarrow \mathbb{N}$
\begin{itemize}
    \item Description : \emph{NetworkGraph.getNearestNode}(GPS coordinate) using the \emph{NetworkGraph} access routine the function returns node number.
\end{itemize}
DijkstraAlg : \emph{NetworkGraph} $\times$ String $\times$ $\mathbb{N} \times \mathbb{N}  \rightarrow seq of \mathbb{N}$
\begin{itemize}
    \item Description : using Dijkstra’s Shortest Path Algorithm given graph with weighted edges, source node, and destination node find shortest path as a seq of nodes.
\end{itemize}
%done shortest route 
\newpage

\section{MIS of \wss{Module Alternative Route}} \label{ModuleSPath} 

\subsection{Template Module}
Alternative Route

\subsection{Uses}%what modules it uses 
Network Graph

\subsection{Syntax}

\subsubsection{Exported Constants}
None.

\subsubsection{Exported Type}
AlternativeRoute = ?

\subsubsection{Exported Access Programs}

\begin{center}
\begin{tabular}{p{2cm} p{4cm} p{4cm} p{2cm}}
\hline
\textbf{Name} & \textbf{In} & \textbf{Out} & \textbf{Exceptions} \\
\hline
\wss{new AlternativeRoute} & NetworkGraph, tuple of (latitude: $\mathbb{R}$, longitude: $\mathbb{R}$), tuple of (latitude: $\mathbb{R}$, longitude: $\mathbb{R}$) & AlternativeRoute & $\mbox{OutOfBoundsCoord}$ \\
\hline
\end{tabular}
\end{center}

\subsection{Semantics}

\subsubsection{State Variables}

\emph{startnode} : $\mathbb{N}$ - graph node nearest the starting GPS coordinate.\\
\emph{endnode} : $\mathbb{N}$ - graph node nearest the destination GPS coordinate.

\subsubsection{Environment Variables}

onlinetransitdatabase : connection to online database to retrieve information layers of bus stops initialised within module to deactivate inaccessible edges on the network graph.

\subsubsection{Assumptions}

\begin{itemize}
    \item Assume the inputted GPS coordinates are valid as they are the output of another module.
\end{itemize}

\subsubsection{Access Routine Semantics}

\noindent \wss{new AlternativeRoute}(graph, startcoord, endcoord):
\begin{itemize}
\item transition: \wss{Set $startnode, endnode, graph := findnode(graph, startcoord)$, $findnode(graph, endcoord) $, \emph{deactivateedges(graph)} at the beginning.} 
\item output: \wss{Creates \emph{AlternativeRoute} using \emp{pathfinder(graph, startcoord, endcoord)}.} 
\item exception: \wss{$(\neg(startnode \in seq of graph.nodes) \Rightarrow \mbox{OutOfBoundsCoord}) \vee (\neg(endnode \in seq of graph.nodes) \Rightarrow \mbox{OutOfBoundsCoord})$} 
\end{itemize}

\subsubsection{Local Functions}

findnode : \emph{NetworkGraph} $\times$ tuple of $(\mathbb{R}, \mathbb{R})  \rightarrow \mathbb{N}$
\begin{itemize}
    \item Description : \emph{NetworkGraph.getNearestNode}(GPS coordinate) using the \emph{NetworkGraph} access routine the function returns node number.
\end{itemize}
pathfinder : \emph{NetworkGraph} $\times$ String $\times$ $\mathbb{N} \times \mathbb{N}  \rightarrow seq of \mathbb{N}$\\
deactivateedges : \emph{NetworkGraph} $\times$ \emph{onlinetransitdatabase} $ \rightarrow NetworkGraph$
\begin{itemize}
    \item Description : deactivate edges inaccessible by bus according to the exported information.
\end{itemize}
%done with alternative path
\newpage

\section{MIS of \wss{Module Route}} \label{ModuleSPath} 

\subsection{Module} 
Route Generation

\subsection{Uses}%what modules it uses 
Shortest Route, Alternative Route, Data Transformation

\subsection{Syntax}

\subsubsection{Exported Constants}
None.

\subsubsection{Exported Type}
None.

\subsubsection{Exported Access Programs}

\begin{center}
\begin{tabular}{p{4cm} p{4cm} p{4cm} p{4cm}}
\hline
\textbf{Name} & \textbf{In} & \textbf{Out} & \textbf{Exceptions} \\
\hline
\wss{GenerateGraph} & seq of tuple of (latitude: $\mathbb{R}$, longitude: $\mathbb{R}$), String: \in \{drive,walk,bike\} & seq of $\mathbb{N}$ & - \\
\hline
\wss{GenerateShortest- Routes} & seq of tuple of (latitude: $\mathbb{R}$, longitude: $\mathbb{R}$), $\mathbb{B}$, String: \in \{time, distance\} & seq of $\mathbb{N}$ & - \\
\hline
\wss{GenerateAlternative- Routes} & seq of tuple of (latitude: $\mathbb{R}$, longitude: $\mathbb{R}$), $\mathbb{B}$ & seq of $\mathbb{N}$ & - \\
\hline
\end{tabular}
\end{center}

\subsection{Semantics}

\subsubsection{State Variables}

\emph{routes} : seq of seq of $\mathbb{N}$ - list of route segments consisting of traces of nodes. The list represent a connected path for a full episode. \\
\emph{graph} : \emph{NetworkGraph} - Created for the given GPS input

\subsubsection{Environment Variables}

None

\subsubsection{Assumptions}

\begin{itemize}
    \item Assume the inputted GPS coordinates are valid as they are the output of another module.
\end{itemize}

\subsubsection{Access Routine Semantics}

\noindent \wss{GenerateGraph}(gpscoords, mode):
\begin{itemize}
\item transition: \wss{Set $graph :=$ \emph{NetworkGraph.new NetworkGraph(getstart(gpscoords), getend(gpscoords), getstops(gpscoords), mode)}}
\item output: \wss{None} 
\item exception: \wss{None} 
\end{itemize}

\noindent \wss{GenerateShortestRoute}(gpscoords, hasstops, weighttype):
\begin{itemize}
\item transition: \wss{Update routes with (hasstops $\Rightarrow$ ($\forall$ connection : tuple | connection $\in$ getconnections(gpscoords) : routes || ShortestRoute.new ShortestRoute(graph, connection[0], connection[1], weighttype)))  | ($\neg$ hasstops $\Rightarrow$ routes || ShortestRoute.new ShortestRoute(graph, getstart(gpscoords), getstops(gpscoords), weighttype))} 
\item output: \wss{None} 
\item exception: \wss{None} 
\end{itemize}

\noindent \wss{GenerateAlternativeRoute}(gpscoords, hasstops):
\begin{itemize}
\item transition: \wss{Update routes with (hasstops $\Rightarrow$ ($\forall$ connection : tuple | connection $\in$ getconnections(gpscoords) : routes || AlternativeRoute.new AlternativeRoute(graph, connection[0], connection[1] )))  | ($\neg$ hasstops $\Rightarrow$ routes || AlternativeRoute.new AlternativeRoute(graph, getstart(gpscoords), getstops(gpscoords) ))} 
\item output: \wss{None} 
\item exception: \wss{None} 
\end{itemize}

\subsubsection{Local Functions}

getstart : seq of tuple of $(\mathbb{R}, \mathbb{R})  \rightarrow tuple of (\mathbb{R}, \mathbb{R})$
\begin{itemize}
    \item Description : returns first coordinate.
\end{itemize}
getend : seq of tuple of $(\mathbb{R}, \mathbb{R})  \rightarrow$ tuple of $(\mathbb{R}, \mathbb{R})$
\begin{itemize}
    \item Description : returns last coordinate
\end{itemize}
getstops : seq of tuple of $(\mathbb{R}, \mathbb{R})  \rightarrow$ seq of tuple of $(\mathbb{R}, \mathbb{R})$
\begin{itemize}
    \item Description : returns list of stop coordinates.
\end{itemize}
getconnections : seq of tuple of $(\mathbb{R}, \mathbb{R})  \rightarrow$ seq of tuple of (tuple of $(\mathbb{R}, \mathbb{R}),$ tuple of $(\mathbb{R}, \mathbb{R})$)
\begin{itemize}
    \item Description : returns list connections of GPS pairs.
\end{itemize}
%done route generation
\newpage

\color{red}
\section{MIS of \wss{Activity Location}} \label{Module} 

\subsection{Template Module}
Activity Location

\subsection{Uses}
None.

\subsection{Syntax}

\subsubsection{Exported Constants}
None

\subsubsection{Exported Type}
ActivityLocation = ?

\subsubsection{Exported Access Programs}

\begin{tabular}{p{2cm} p{3cm} p{4cm} p{4cm}}
\hline
\textbf{Name} & \textbf{In} & \textbf{Out} & \textbf{Exceptions} \\
\hline
\wss{new ActivityLocation} & name: String, lat: Latitude ($\mathbb{R}$), lon: Longitude ($\mathbb{R}$) &  Activity Location} \\
\hline

\end{tabular}

\subsection{Semantics}

\subsubsection{State Variables}
\emph{name} : String of Activity Location name\\
\emph{lat} : latitude of Activity Location ($\mathbb{R}$)\\
\emph{lon} : longitude of Activity Location ($\mathbb{R}$)

\subsubsection{Environment Variables}
None.
\subsubsection{Assumptions}
\begin{itemize}
    \item Assume the values for latitude, longitude and name provided are valid
\end{itemize}

\subsubsection{Access Routine Semantics}
None
\color{black}

\newpage
\color{red}
\section{MIS of \wss{Stop Point}} \label{Module} 

\subsection{Template Module}
Stop Point

\subsection{Uses}
None.

\subsection{Syntax}

\subsubsection{Exported Constants}
None

\subsubsection{Exported Type}
StopPoint = ?

\subsubsection{Exported Access Programs}

\begin{tabular}{p{2cm} p{3cm} p{4cm} p{4cm}}
\hline
\textbf{Name} & \textbf{In} & \textbf{Out} & \textbf{Exceptions} \\
\hline
\wss{new StopPoint} & lat: Latitude ($\mathbb{R}$), lon: Longitude ($\mathbb{R}$) &  StopPoint} \\
\hline

\end{tabular}

\subsection{Semantics}

\subsubsection{State Variables}
\emph{lat} : latitude of Stop Point ($\mathbb{R}$)\\
\emph{lon} : longitude of Stop Point ($\mathbb{R}$)\\

\subsubsection{Environment Variables}
None.
\subsubsection{Assumptions}
\begin{itemize}
    \item Assume the values for latitude, longitude  provided are valid
\end{itemize}

\subsubsection{Access Routine Semantics}
None
\color{black}
\newpage

\section{MIS of \wss{Generating Activity Locations}} \label{ModuleSPath} 
%\wss{Use labels for cross-referencing}

\subsection{Module}
Fetch Activity Locations
\subsection{Uses}%what modules it uses 
\textcolor{red}{\sout{Uses no other modules}}
\textcolor{red}{Activity Location, Stop Point}

%make overpy an environemental variable and just name it as online database
%don't use overpy

\subsection{Syntax}

\subsubsection{Exported Constants}
None.

\subsubsection{Exported Access Programs}

\begin{center}
\begin{tabular}{p{3cm} p{4cm} p{4cm} p{4cm}}
\hline
\textbf{Name} & \textbf{In} & \textbf{Out} & \textbf{Exceptions} \\
\hline
\wss{fetchStopAL} & \textcolor{red}{collection of Stop Points}  & \textcolor{red}{collection of tuples (Stop Point, Activity Location)} & EmptyListException \\
\hline
\end{tabular}
\end{center}

\subsection{Semantics}

\subsubsection{State Variables}

\wss{Not applicable}

\subsubsection{Environment Variables}

\wss{onlinenetworkdatabase: connection to online database to retrieve information layers of
intersections, roads, and paths, and activity locations.}
\subsubsection{Assumptions}

\begin{itemize}
\item Collection of stop locations longitudes and latitudes that are correct and valid
\item Tolerance of 25 meter radius is appropriate for finding activity locations
\item Onlineworkdatabase is accurate and accessible within 10 minutes
\end{itemize}
\subsubsection{Access Routine Semantics}

\noindent \wss{fetchStopAL(listOfStops)}():
\begin{itemize}
\item transition: Not applicable
% \wss{For all stops in list\_of\_stops, calls local auxiliary function fetchActivityLocation() by sending in latitude and longitude parameters to find list of activity locations per stop}
\item output: \textcolor{red}{collection of tuples (Stop Point, Activity Location)}
\item exception: \wss{$listOfStops = \{\} \Rightarrow \mbox{EmptyListException}$} 
\end{itemize}

\subsubsection{Local Functions}
fetchActivityLocations : tuple of (latitude: $\mathbb{R}$, longitude: $\mathbb{R}$)
\begin{itemize}
    \item Description: Computes activity location given the latitude and longitude of a specific stop location by fetching activity locations in a 25 meter radius from onlineworkdatabase and \textcolor{red}{\sout{returns list of activity location names and latitudes and longitudes} returns list of ActivityLocation objects} 
\end{itemize}
  
\newpage

\section{MIS of \wss{Module Mapping}} \label{ModuleMapping} 

\subsection{Template Module}
Mapping

\subsection{Uses}%what modules it uses 
Network Graph, Shortest Route, Alternative Route

\subsection{Syntax}

\subsubsection{Exported Constants}
None.

\subsubsection{Exported Type}
Display = ?

\subsubsection{Exported Access Programs}

\begin{tabular}{p{3cm} p{4cm} p{4cm} p{3cm}}
\hline
\textbf{Name} & \textbf{In} & \textbf{Out} & \textbf{Exceptions} \\
\hline
\wss{new Mapping} & \emph{NetworkGraph}, seq of \emph{ShortestRoute}, seq of \emph{AlternativeRoute}, seq of $\mathbb{N}$, seq of $\mathbb{N}$ & HTML FILE & - \\
\hline
\wss{updateMapping} & seq of \emph{ShortestRoute}, seq of \emph{AlternativeRoute}, seq of $\mathbb{N}$, seq of $\mathbb{N}$ & HTML FILE & - \\
\hline
\end{tabular}


\subsection{Semantics}

\subsubsection{State Variables}

\emph{map} : HTML FILE - displays mapped routes and points. 

\subsubsection{Environment Variables}

IOmanager : it is the access point between the local memory and the application. It is used when saving and updating HTML files locally. 

\subsubsection{Assumptions}

\begin{itemize}
    \item Assume all the inputs are valid as they are outputs of previous modules. 
\end{itemize}

\subsubsection{Access Routine Semantics}

\noindent \wss{new Mapping}(graph, shortest, alternative, activitylocations, episodes):
\begin{itemize}
\item transition: \wss{None.} 
\item output: \wss{Create a \emph{Mapping} object and saves it locally with \emph{IOmanager}} 
\item exception: \wss{None} 
\end{itemize}

\noindent \wss{updateMapping}(shortest, alternative, activitylocations, episodes):
\begin{itemize}
\item transition: \wss{Add new elements to \emph{Mapping} object and saves it locally with \emph{IOmanager}} 
\item output: \wss{None.} 
\item exception: \wss{None} 
\end{itemize}

\subsubsection{Local Functions}

getNode : seq of tuples of $(\mathbb{R}, \mathbb{R}) \Rightarrow $ seq of $\mathbb{N}$
\begin{itemize}
    \item Description: converts GPS coordinates to graph nodes. 
\end{itemize}

%done moving module
\newpage

\section{MIS of \wss{Main Module}} \label{Module}

\subsection{Module}

\wss{Main}

\subsection{Uses}
Input Processing Module, Episode Generation and Mode Detection Module, Activity Location Module, Route Generation Module, Data Transformation Module
\subsection{Syntax}

\subsubsection{Exported Constants}
N/A
\subsubsection{Exported Access Programs}

\begin{center}
\begin{tabular}{p{4cm} p{4cm} p{4cm} p{4cm}}
\hline
\textbf{Name} & \textbf{In} & \textbf{Out} & \textbf{Exceptions} \\
\hline
\wss{inputProcessing} & $\mathbb{N}$, seq of String & .SHP files & InvalidCoord \\
\hline
\wss{episodeGeneration- ModeDetection} & CSV file & CSV file & InvalidFile \\
\hline
\wss{findActivityLocations} & CSV file & CSV file & EmptyListException \\
\hline
\wss{generateGraph} & CSV file, String $\mathbb{B}$ & $\mathbb{B}$ & \mbox{InvalidMode}, \mbox{OutOfBoundsCoord}\\
\hline
\wss{generateShortestPath} & CSV file, String $\mathbb{B}$ & seq of $\mathbb{N}$ & $\mbox{InvalidWeight}$, $\mbox{OutOfBoundsCoord}$\\
\hline
\wss{generateAlternative- Path} & CSV file $\mathbb{B}$ & seq of $\mathbb{N}$ & $\mbox{OutOfBoundsCoord}$ \\
\hline
\wss{mapEpisodes} & NetworkGraph, CSV file & HTML file & - \\
\hline
\wss{mapActivityLocations} & NetworkGraph, CSV file & HTML file & - \\
\hline
\wss{mapSRoute} & NetworkGraph, CSV file & HTML file & - \\
\hline
\wss{mapARoute} & NetworkGraph, CSV file & HTML file & - \\
\hline

\end{tabular}
\end{center}

\newpage


\bibliographystyle {plainnat}
\bibliography{MIS.bib}

\newpage

\section{Appendix} \label{Appendix}

\wss{Extra information if required}

\end{document}
