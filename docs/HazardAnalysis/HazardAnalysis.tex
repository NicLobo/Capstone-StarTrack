\documentclass{article}

\usepackage{booktabs}
\usepackage{tabularx}
\usepackage[letterpaper, margin=1in]{geometry}
\usepackage{fixltx2e}
\usepackage{longtable}
\usepackage{placeins}
\usepackage{xr}
\usepackage{amssymb}
\usepackage{pdflscape}
\usepackage{multicol}
\usepackage{multirow}
\usepackage{verbatim}
\usepackage{rotating}
\usepackage{pdfpages}
\usepackage{lscape}
\usepackage{hyperref}
\usepackage{float}
\usepackage{graphicx}
\usepackage{makecell}

\usepackage{xcolor}
\usepackage{ulem}

\newcounter{Sreqnum} %Safety Requirement Number

\title{\textbf{yoGERT GIS Toolbox}\\ Capstone 4G06\\ Hazard Analysis}

\author{Team 19,
		\\ Smita Singh, Abeer Alyasiri, Niyatha Rangarajan,\\ Moksha Srinivasan, Nicholas Lobo, Longwei Ye \\\\
}

\date{\today}


\input{}

\begin{document}
\nocite{*}
\maketitle

\newpage

%Page number should be in arabic order
\pagenumbering{arabic}
\tableofcontents
\listoftables
\listoffigures

\section{Revisions}

\begin{table}[hp]
\caption{Revision History} \label{TblRevisionHistory}
\begin{tabularx}{\textwidth}{llX}
\toprule
\textbf{Date} & \textbf{Developer(s)} & \textbf{Change}\\
\midrule
Date1 & Name(s) & Description of changes\\
10/19/2022 & Niyatha Rangarajan & Failure Modes and Effects Analysis\\
10/18/2022 & Smita Singh & Safety and Security Requirements\\
10/13/2022 & Moksha Srinivasan & Sections 3-8.1\\
10/18/2022 & Abeer Alyasiri & Safety and Security Requirements, Roadmap\\
10/18/2022 & Longwei Ye & Failure Modes and Effects Analysis\\
10/19/2022 & Nicholas Lobo & Failure Modes and Effects Analysis\\
11/22/2022 & Longwei Ye & Format changes based on feedback\\
11/23/2022 & Longwei Ye & Update FEMA Table\\
\bottomrule
\end{tabularx}
\end{table}

\section{Definitions}
\subsection{Naming Conventions and Terminology}
\begin{tabular}{l p{6cm}} 
  \toprule		
  \textbf{symbol} & \textbf{description}\\
  \midrule 
  GPS & Global Positioning Systems.\\
  GIS & Geographical Information Systems.\\
  MIS & Module Interface Specification.\\
  GUI & Graphical User Interface. \\
  CSV/.csv & Comma Separated Values is a file type that contains large amounts of data separated by commas.\\
  SRS & System Requirement Specification.\\
  GB & Gigabyte \\

  \bottomrule
\end{tabular}\\

\section{Introduction}
This document is the hazard analysis of the yoGERT GIS toolbox. The toolbox is a software that aids in deriving insights from transportation data. This includes route choice estimations, travel mode detection, and travel episode identification/analysis.
\section{Scope and Purpose of Hazard Analysis}
The scope of this document is to identify possible hazards within the yoGERT system, the causes and effects of failure, steps for mitigation, as well as safety and security requirements.
\section{System Boundary}
The system referred to in this document that the hazard analysis is conducted upon consists of:
\begin{itemize}
    \item The toolbox (installed on users' personal computers) made up of the following major components:
    \begin{itemize}
        \item Data Pre-processing
        \item Travel Episode Identification
        \item Travel Mode Detection
        \item Activity Location Identification
        \item Route Choice Analysis
        \item Visualization Module
    \end{itemize}
    \item The user's personal computer
    
\end{itemize}
\section{Definition of the Hazard}
The definition of a hazard is from Nancy Leveson's work as follows: A property or condition in the system along with a condition in the environment that has the potential to cause harm or damage. In yoGERT, there are hazards with regard to safety (to data preservation) and security (restricted data access). \textcolor{blue}{Besides, hazards can be also found due to human-errors(wrong actions to the system).}
\section{Critical Assumptions}
One critical assumption is regarding the system boundary. For example, if the stretch goal of adding a GUI and GIS software plug-in is completed, the hazard analysis will have to be extended. Another critical assumption is that the user knows how to upload a file using a file path. \\

\section{Failure Modes and Effects Analysis}
\subsection{Hazards Out of Scope}
The hazards out of scope would be:
\begin{itemize}
    \item \sout{The user's personal computer}
    \item The validity of input data
\end{itemize}

\noindent The \sout{user's personal computer and} validity of input data are integral parts of our system's correctness. It is out of scope to develop a mechanism to verify that data is correct/applicable, and hence hazards with respect to correctness and validity of data are out of scope. To ensure that we can process the user's data we will be sanitizing and normalizing data before applying functions. The user's personal computer is managed by the user and software updates are applied by the manufacturer, hence we cannot control all hazards. 

\newpage 
\begin{landscape}
\subsection{Failure Modes & Effects Analysis Table}

%%%divide table by those sections maybe?
System: GERT toolbox\\
Subsystem:N/A \\
Phase/Mode:System Requirements\\
\begin{table}[H]
\centering
\scalebox{0.75}{

\begin{tabular}{|p{1.8 in}|p{2.0 in}|p{2 in}|p{2 in}|p{2.3 in}|p{0.5 in}|p{0.5 in}|} \hline
		\textbf{Design Function} & \textbf{Failure Modes} & \textbf{Effects of Failure} & \textbf{Causes of Failure} & \textbf{Recommended Actions} & \textbf{SR} & \textbf{Ref.}\\\hline


\multirow{3}{1.8in}{\textcolor{blue}{Storing GPS points through internet connection.}\sout{Storage of users geo-location.}} 

& User's location data is corrupted  
& Unable to use program without clearing out corrupted data
& a. Bad data inputted by user\newline
  b. Corrupted files being load into system\newline
& a. Data is checked and sanitized before being added into system\newline
& a:SR\ref{IR4} %6
& H1-1\label{H1-1} \\ 
 
& User's location data is not saved
& User must input data again
&  a. Disconnection from internet\newline
   b. Computer crash while using software \newline
& 
  b. Last actions made by user in the software are saved  
& a:SR\ref{IR2}\newline b:SR\ref{PR1}%8
& H1-2 \\ \hline
 

 

\multirow{3}{1.8in}{\textcolor{blue}{Handling large datasets}\sout{Memory usage of the system} }
& 
Unprocessed csv file for gps input is too large   
& 
Unable to process the gps data for other functions 
& a. Incomplete system functionality\newline
  b. File if produced might be inaccurate or skewed\newline
  c. Uncertainity in output \newline 
 
& a. State file size restrictions for the system \newline
& a. SR\ref{IR5} \newline 
& H2-1\\



& Processed gps data is too large to be stored in the output directory
& Unable to use the system for creating episodes, routes and activities
& a. Useful data manipulation can be attained\newline
  b. The system is incomplete in functionality\newline
  c. Benefits of analysing episodes is prevented \newline 
& b. Produced file can be compressed to meet storage requirements\newline
& b. SR\ref{IR5} \newline 
& H2-2 \\ \hline
 


\multirow{3}{1.8in}{\textcolor{blue}{Inconsistency in data results}\sout{General use of the system}} 
& Program crashes unexpectedly
& Program is terminated, unable to continue current progress, data of the
current progress could be lost
& 
a. Crash due to lack of supported libraries\newline
b. User device runs out of power\newline
& 
a. Check if the supported libraries are installed at the start of the program\newline
\textcolor{blue}{b. Charge the device so that the system can work again\newline}
& SR\ref{ACR2}%2
& H3-1\label{H3-1} \\


& Output does not meet the desired accuracy
& Unusable output, unable to proceed to the next step
&
a. Input data is in wrong format
b. Input file is corrupted
& 
a. Same as H1-2 \newline
b. Validating against requirements and mathematical specifications\newline
& SR\ref{IR3}\newline SR\ref{IR1} \newline SR\ref{ADR1}
& H3-2\label{H3-2}  \\



%& Data is leaked to other users %%%not possible with our local toolbox
%& User's work could be plagiarized by other users
%&
%a. The exists of system vulnerability\newline
%& 
%a. All data is stored locally\newline %%%our scope project is already defined locally.
%b. Output data can be encrypted with password\newline
%& SR7
%& H2-3 \\ \hline

& \sout{An unauthorized action} \textcolor{blue}{Actual output does not meet the user's expectation at all}
& Data can be corrupted and the produce wrong output 
&
a. \textcolor{blue}{An unauthorized or an unexpected}\sout{Unexpected} action made by the user\newline
& 
a. Revert the changes made by the unauthorized action\newline
& SR\ref{ACR1}, SR\ref{ADR1}
& H3-3 \\ \hline



 

\end{tabular}
}
\caption{FMEA TABLE}
\label{Table:FMEA TABLE}
\end{table}
\end{landscape}


\section{Safety and Security Requirements}
The section lists the newly formulated requirements. The new requirements will be added to the SRS document and the previous requirements will be removed before Revision 1 milestone.

\begin{itemize} %for all safety items
\subsection{Access Requirements}

\item[SR\refstepcounter{Sreqnum}\theSreqnum
% \label{ACR}:] \plt{\emph{The system must ensure that the user session is password protected. }}
\label{ACR1}:] \plt{The system must allow users to have access to \textcolor{blue}{read and modify} the data they've uploaded}
\begin{itemize}
    \item \textbf{Rationale:} This would allow users to edit inputted data and make any necessary changes. If a problem occurs, the user should be able to retrieve their data instead of having to restart the process or software which would be time consuming
    \item \textbf{Associated Hazards: } H3-3 (\ref{H1-1})
\end{itemize}

\item[SR\refstepcounter{Sreqnum}\theSreqnum
\label{ACR2}:] \plt{The system shall allow access to all system services and data outputs.}
\begin{itemize}
    \item \textbf{Rationale:} This is the main objective to the application to satisfy user goals. If a problem occurs, the system will be completely ineffective and not workable. 
    \item \textbf{Associated Hazards:} H3-1 (\ref{H1-1})
\end{itemize}

\subsection{Integrity Requirements}
\item[SR\refstepcounter{Sreqnum}\theSreqnum
% \label{IR}:] \plt{\emph{The system shall encyrpt user information. }}
\label{IR1}:] \plt{The system shall output correct calculated or modified data}
\begin{itemize}
    \item \textbf{Rationale:} The user should not have to question the accuracy of the data outputted. If the data is not accurate or correctly calculated it is contrary to the goal of the system. 
    \item \textbf{Associated Hazards:} H3-3 (\ref{H1-1})
\end{itemize}

\item[SR\refstepcounter{Sreqnum}\theSreqnum
\label{IR2}:] \plt{The system will only modify necessary data}
\begin{itemize}
    \item \textbf{Rationale:} The system would be wasting resources and time if any other modification or unnecessary calculations occur. It would also be unethical to use the data in a way that the user is unaware of and has not consented to. 
    \item \textbf{Associated Hazards: } H3-3 (\ref{H1-1})
\end{itemize}

\item[SR\refstepcounter{Sreqnum}\theSreqnum
\label{IR3}:] \plt{The system shall produce accuracy of 80\% when graphing location data points on a map.}
\begin{itemize}
    \item \textbf{Rationale:} The graphics produced to the system should be consistent to produce valuable information to be reused. When the graphic points deviate from the actual location the system will be counterproductive and user will be alarmed.  
    \item \textbf{Associated Hazards:} H3-2 (\ref{H1-1})
\end{itemize}

\item[SR\refstepcounter{Sreqnum}\theSreqnum
\label{IR4}:] \plt{The system shall provide warning log messages of \sout{proper or expected} \textcolor{blue}{improper or unexpected} system uses. }
\begin{itemize}
    \item \textbf{Rationale:} The system should provide helpful information when navigating around the system. This ensures a mitigation response to unexpected user activity.
    \item \textbf{Associated Hazards:}  H1-2 (\ref{H1-1})
\end{itemize}

\item[SR\refstepcounter{Sreqnum}\theSreqnum
\label{IR5}:] \plt{The system shall confirm upload and output files are not larger than 1GB.}
\begin{itemize}
    \item \textbf{Rationale:} The system needs to provide secure methods of handling system files. This ensures safe saving and uploading of files. 
    \item \textbf{Associated Hazards:} H2-1, H2-2 (\ref{H1-1})
\end{itemize}


\subsection{Privacy Requirements}
\item[SR\refstepcounter{Sreqnum}\theSreqnum
\label{PR1}:] \plt{The system will only store data locally.}
\begin{itemize}
    \item \textbf{Rationale:} The data does not have to be hosted on a remote location such as online database; which would mean better security and protection for the user's data. Failure to store makes the system unresponsive and idle. 
    \item \textbf{Associated Hazards:} H3-4 (\ref{H1-1})
\end{itemize}

\subsection{Audit Requirements}
\item[SR\refstepcounter{Sreqnum}\theSreqnum
\label{ADR1}:] \plt{The system should be verifiable against the requirements and MIS }
\begin{itemize}
    \item \textbf{Rationale:} The system must be verified with logical and deductible methods. Failure to meet the requirement results with poor demonstration of requirements.
    \item \textbf{Associated Hazards:} H3-1, H3-2, H3-3 (\ref{H1-1})
\end{itemize}
\textcolor{blue}{
\item[SR\refstepcounter{Sreqnum}\theSreqnum
\label{ADR1}:] \plt{The system should pass audit requirements for producing correct and reliable data. }
\begin{itemize}
    \item \textbf{Rationale:} The system must be verified with logical and deductible methods. Failure to meet the requirement results with poor demonstration of requirements.
    \item \textbf{Associated Hazards:} H3-1, H3-2, H3-3 (\ref{H1-1})
\end{itemize}}

\subsection{Immunity Requirements}
N/A

\end{itemize}

\section{Roadmap}
%\wss{Which safety requirements will be implemented as part of the capstone $timeline?
%Which requirements will be implemented in the future?}

The document outlined new safety and security requirements to mitigate and avoid system hazards. The project aims to implement all the safety requirements. However, a priority guideline has been decided that leaves SR\ref{IR2}, SR\ref{IR3}, and SR\ref{IR4} requirements to be more likely than 50\% implemented in the future due to constrained capstone timeline. 

\bibliographystyle{IEEEtran}
\bibliography{hazardanalysis}

\end{document}
