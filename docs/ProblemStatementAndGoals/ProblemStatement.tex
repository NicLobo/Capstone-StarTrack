\documentclass{article}

\usepackage{tabularx}
\usepackage{booktabs}
\usepackage{array,multirow}

\title{Problem Statement and Goals\\\progname}

\author{\authname}

\date{}

%% Comments

\usepackage{color}

\newif\ifcomments\commentstrue %displays comments
%\newif\ifcomments\commentsfalse %so that comments do not display

\ifcomments
\newcommand{\authornote}[3]{\textcolor{#1}{[#3 ---#2]}}
\newcommand{\todo}[1]{\textcolor{red}{[TODO: #1]}}
\else
\newcommand{\authornote}[3]{}
\newcommand{\todo}[1]{}
\fi

\newcommand{\wss}[1]{\authornote{blue}{SS}{#1}} 
\newcommand{\plt}[1]{\authornote{magenta}{TPLT}{#1}} %For explanation of the template
\newcommand{\an}[1]{\authornote{cyan}{Author}{#1}}

%% Common Parts

\newcommand{\progname}{ProgName} % PUT YOUR PROGRAM NAME HERE
\newcommand{\authname}{Team \#, Team Name
\\ Student 1 name and macid
\\ Student 2 name and macid
\\ Student 3 name and macid
\\ Student 4 name and macid} % AUTHOR NAMES                  

\usepackage{hyperref}
    \hypersetup{colorlinks=true, linkcolor=blue, citecolor=blue, filecolor=blue,
                urlcolor=blue, unicode=false}
    \urlstyle{same}
                                


\begin{document}

\maketitle

\begin{table}[hp]
\caption{Revision History} \label{TblRevisionHistory}
\begin{tabularx}{\textwidth}{llX}
\toprule
\textbf{Date} & \textbf{Developer(s)} & \textbf{Change}\\
\midrule
Sept. 23, 2022 & Nicholas Lobo & Problem Statement\\\\
Date2 & Name(s) & Description of changes\\
... & ... & ...\\
\bottomrule
\end{tabularx}
\end{table}

\section{Problem Statement}



\subsection{Problem}
With the large usage of global positioning systems, software has been created to help researchers develop tools and methods based off peoples GPS data. 
One such software is the ArcPro toolbox that can match GPS traces to transportation networks. This software suffers from specific data requirements that limit usability, 
limited functionality and is not able to handle the use of larger GPS data sets. In order to overcome these weaknesses the ArcPro toolbox must be re-engineered with a focuses 
on transferability, modularity, and scalability as well as being open source without using any proprietary software. 

\subsection{Inputs and Outputs}
\begin{table}[h]
    \centering
    \begin{tabular}{|p{6cm}|p{6cm}|}
    \hline
    Input & Output  \\
    \hline
    \multirow{2}{5cm}{A dataset of latitude and longitude positions and times based on persons typical route during the day} & Identify sections of each travelers day based on the location they are in for some period of time.  \\\cline{2-2} 
    & Identify the travelers method of transportation they are using during each episode(i.e bus, walking,)  \\\cline{2-2} 
    & Estimate the possible route a traveler will take\\\cline{2-2} 
    \hline
    \end{tabular}
\end{table}

\subsection{Stakeholders}

\subsection{Environment}

\wss{Hardware and software}

\section{Goals}

\section{Stretch Goals}

\end{document}